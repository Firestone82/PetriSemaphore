\section*{Analýza sítě}
\label{sec:network_analyse}

V této sekci se budeme zabývat analýzou Petriho sítě, která reprezentuje chování semaforů na křižovatce.

\subsection*{P-Invarianty Petriho sítě}
\label{subsec:p_invariants}

\newcommand{\pI}  {0, 0, 0, 0, 0, 0, 1, 1, 0, 0, 1, 0, 0}
\newcommand{\pII} {0, 0, 1, 1, 0, 0, 0, 0, 0, 0, 0, 0, 1}
\newcommand{\pIII}{0, 1, 0, 0, 1, 0, 0, 0, 0, 0, 0, 0, 1}
\newcommand{\pIV} {0, 0, 0, 0, 0, 0, 1, 0, 1, 1, 0, 0, 0}
\newcommand{\pV}  {1, 0, 0, 0, 0, 1, 0, 0, 0, 0, 0, 1, 0}
\newcommand{\pAll}{1, 1, 1, 1, 1, 1, 2, 1, 1, 1, 1, 1, 2}

Pro tuto síť existuje celkem 5 p-invariantů vypsaných níže a výsledný p-invariant po sečtení všech hodnot je $Y_6$ = \orderArrayNumbered{\pAll}.\@
O všech p-invariantech můžeme říci, že jsou \textbf{minimální}, \textbf{netriviální} a \textbf{v kanonickém tvaru}.
A celkově tento systém p-invariantů je \textbf{úplný}.
Počáteční značení $M_0$ = (\orderArrayNumbered{\mO}).

\begin{table}[h!]
    \renewcommand{\arraystretch}{1.5}
    \resizebox{\textwidth}{!}{%
        \begin{tabular}{p{1cm} p{8cm} p{6cm}}%
            \ & \ul{Strukturální}                    & \ul{Systémové} \\
            \ & $Y_1^T$ - \orderArrayNumbered{\pI}   & $Y_1^T$ \cdot $M^0$ = 1 \\
            \ & $Y_2^T$ - \orderArrayNumbered{\pII}  & $Y_2^T$ \cdot $M^0$ = 1 \\
            \ & $Y_3^T$ - \orderArrayNumbered{\pIII} & $Y_3^T$ \cdot $M^0$ = 1 \\
            \ & $Y_4^T$ - \orderArrayNumbered{\pIV}  & $Y_4^T$ \cdot $M^0$ = 1 \\
            \ & $Y_5^T$ - \orderArrayNumbered{\pV}   & $Y_5^T$ \cdot $M^0$ = 3 \\
        \end{tabular}
    }
    \label{tab:p_invariants}
\end{table}
\vspace{-1em}

\subsection*{T-Invarianty Petriho sítě}
\label{subsec:t_invariants}

\newcommand{\tI}  {1, 1, 0, 0, 0, 0, 0, 0, 0, 0, 0, 0}
\newcommand{\tII} {0, 0, 0, 0, 0, 0, 0, 0, 0, 0, 1, 1}
\newcommand{\tIII}{0, 0, 1, 0, 0, 1, 0, 1, 1, 0, 0, 0}
\newcommand{\tIV} {0, 0, 0, 1, 1, 0, 1, 0, 0, 1, 0, 0}
\newcommand{\tAll}{1, 1, 1, 1, 1, 1, 1, 1, 1, 1, 1, 1}

Stejně tak zde nalezneme 4 t-invarianty, jejichž součet dává výsledný t-invariant \\$X_5$ = (\tAll).\@
Tyto invarianty jsou \textbf{minimální}, \textbf{netriviální} a \textbf{v kanonickém tvaru}.
Celkově tento systém t-invariantů je \textbf{úplný}.

\begin{table}[h!]
    \renewcommand{\arraystretch}{1.5}
    \resizebox{\textwidth}{!}{%
        \begin{tabular}{p{1cm} p{8cm} p{6cm}}%
            \ & \ul{Strukturální}   & \ul{Systémové} \\
            \ & $X_1^T$ = ({\tI})                                                    & $\sigma_1$ = $t_A$, $t_B$                \\[-5pt]
            \ & {\footnotesize pro $M^0$ = (3, 0, 1, 0, 0, 0, 0, 1, 0, 0, 1, 0, 1)}  &                                          \\[5pt]
            \ & $X_2^T$ = ({\tII})                                                  & $\sigma_1$ = $t_L$, $t_K$                \\[-5pt]
            \ & {\footnotesize pro $M^0$ = (3, 0, 0, 0, 1, 0, 1, 0, 1, 0, 0, 0, 0)}  &                                          \\[5pt]
            \ & $X_3^T$ = ({\tIII})                                                   & $\sigma_1$ = $t_F$, $t_C$, $t_I$, $t_H$  \\[-5pt]
            \ & {\footnotesize pro $M^0$ = (3, 0, 0, 0, 1, 0, 1, 0, 1, 0, 0, 0, 0)}  &                                          \\[5pt]
            \ & $X_4^T$ = ({\tIV})                                                   & $\sigma_1$ = $t_G$, $t_J$, $t_D$, $t_E$  \\[-5pt]
            \ & {\footnotesize pro $M^0$ = (3, 0, 1, 0, 0, 0, 0, 1, 0, 0, 1, 0, 1)}  &                                          \\
        \end{tabular}
    }
    \label{tab:t_invariants}
\end{table}
\vspace{-1em}

\newpage
\subsection*{Vlastnosti PN systému}
\label{subsec:pn_properties}

Za pomocí P a T invariantů můžeme o síti říci tyto vlastnosti:

\begin{itemize}
    \item Síť je \textbf{konzervativní}, protože existuje p-invariant pokrývající všechny stavy. \\ $Y_6$ = \orderArrayNumbered{\pAll}
    \item Síť je \textbf{\ut{striktně} repetiční} protože existuje t-invariant pokrývající všechny přechody. \\ $X_5$ = (\tAll)
    \item Síť je \textbf{k-omezená} pro k = 3, kde k je maximální počet aut pohybující se v křižovatce.
    \item {
        Síť je \textbf{bezpečná} pouze pro vnitřní komponentu systému křižovatky.
        Celý systém je \textbf{bezpečný} v momentě, kdy je síť omezena pro maximálně 1 auto.
    }
    \item Síť je \textbf{živá}, vzhledem k vytvořenému grafu dosažitelnosti pro výchozí značení.
    \item Síť je \textbf{reverzibilní}, protože dle grafu dosažitelnosti neexistuje stav, ze kterého by se nedalo dostat zpět do výchozího značení.
\end{itemize}

\endinput