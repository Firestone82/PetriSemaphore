\section*{Závěr}
\label{sec:summary}

Vytvořená Petriho síť přesně modeluje chování semaforů na křižovatce, čímž poskytuje robustní základ pro analýzu a návrh systémů řízení dopravního toku.
Simulace ověřily, že model zajišťuje bezpečné a spolehlivé střídání světelných signálů v obou směrech provozu.

Navržený model je připraven k dalšímu rozvoji, například k začlenění adaptivního řízení na základě aktuálního dopravního zatížení nebo integraci přechodů pro chodce.
Během analýzy bylo zjištěno, že při zvýšení maximálního počtu vozidel na křižovatce dochází k expanzivnímu růstu grafu dosažitelnosti (viz \tabref{tab:reachability_count}), což však nemá negativní dopad na funkčnost nebo vlastnosti samotné sítě.

\endinput