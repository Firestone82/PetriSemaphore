\section*{Návrh sítě}
\label{sec:network_design}

Pro popis jednotlivých stavu a přechodu na obrázku \figref{fig:network} bude využita zkratka $p_i$ pro místa a $t_i$ pro přechody, kde $i$ je identifikátor před symbol dvojtečkou.

\begin{figure}[h!]
    \centering
    \includegraphics[width=1.0\textwidth]{assets/pes_network}
    \caption{Petriho síť reprezentující chování semaforů na křižovatce.}
    \label{fig:network}
\end{figure}

% Table of transitions
\begin{table}[h!]
    \centering
    \resizebox{0.7\textwidth}{!}{
        \begin{tabular}{c|c|c|c|c|c|c|c|c|c|c|c|c|}
            \textbf{} & \textbf{$t_a$} & \textbf{$t_b$} & \textbf{$t_c$} & \textbf{$t_d$} & \textbf{$t_e$} & \textbf{$t_f$} & \textbf{$t_g$} & \textbf{$t_h$} & \textbf{$t_i$} & \textbf{$t_j$} & \textbf{$t_k$} & \textbf{$t_l$} \\
            \hline
            %                   A    B    C    D    E    F    G    H    I    J    K    L
            \textbf{$p_1$}   & -1 & +1 &    &    &    &    &    &    &    &    & +1 & -1 \\
            \textbf{$p_2$}   & +1 & -1 &    &    &    &  0 &    &    &    &    &    &    \\
            \textbf{$p_3$}   &    &  0 &    &    & +1 &    & -1 &    &    &    &    &    \\
            \textbf{$p_4$}   &    &    &    & +1 & -1 &    & +1 &    &    & -1 &    &    \\
            \textbf{$p_5$}   &    &    &    &  0 & -1 &    &    &    &    & +1 &    &    \\
            \textbf{$p_6$}   &    &    &    &    &    &    & +1 &    &    & -1 &    &    \\
            \textbf{$p_7$}   &    &    &    & -1 &    &    &    &    &  0 & +1 &    &    \\
            \textbf{$p_8$}   &    &    &    &    &    &    &  0 &    &    &    & -1 & +1 \\
            \textbf{$p_9$}   &    &    &    &    &    & -1 &    & +1 &    &    &  0 &    \\
            \textbf{$p_1_0$} &    &    & -1 &    &    & +1 &    & -1 & +1 &    &    &    \\
            \textbf{$p_1_1$} &    &    & +1 &    &    &    &    & -1 &  0 &    &    &    \\
            \textbf{$p_1_2$} &    &    & -1 &    &    & +1 &    &    &    &    &    &    \\
            \textbf{$p_1_3$} &    &    & +1 &  0 &    &    &    &    & -1 &    &    &    \\
        \end{tabular}
    }
    \caption{Matice incidence Petriho sítě. \footnotemark}
    \label{tab:transitions}
\end{table}

\footnotetext[1]{Hodnoty 0 představují obousměrnou výměnu tokenu mezi místem. Hodnoty +1 a -1 značí přidání nebo odebrání tokenu z místa.}

\endinput